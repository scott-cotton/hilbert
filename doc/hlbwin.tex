\documentclass{article}
\author{Scott Cotton\\
IRI France SAS\\
wsc@iri-labs.com}
\title{Windowing The Hilbert Transform}
\begin{document}
\maketitle
\abstract{
The Hilbert transform provides a method for rotating a signal
$90^{\circ}$ in all frequencies.  This transform is useful for
envelope tracking and decomposition of a signal into a single
amplitude and frequency modulated wave.  Hilbert transforms are
simple to implement via the Fourier transform, with one exception: 
the edge effects.  

In this document, we present a simple windowing method for constructing
a Hilbert transformed signal smoothly from a series of fast Fourier transforms
FFTs.  Traditional windowing methods modulate the edge effects, spreading them
out (albeit smoothly).  The windowing method presented here gives better results than 
FIR filtering and traditional windowing, and moreover provides a solid basis
for signal representation by instantaneous frequency/amplitude modulation, as
phase differences and amplitude envelope are by construction coherent accross block 
boundaries.
}

\section{Introduction}
The Hilbert transform is a Fourier based transform which can be seen as a 
filter of "negative" frequencies or, alternatively, a method of producing 
an all-pass, phase altered signal in which each frequency in the frequency
domain is rotated by $\frac{\pi}{2}$, a quarter cycle.  Such a rotation 
puts a signal in {\em quadrature}, making it easy to represent in terms of 
amplitude and phase.  Hilbert transforms are usually applied in narrow band
contexts because of analytic difficulties of understanding the meaning of
instaneous frequency in a wide band context.  However, Hilbert transforms also
provide a nice way of defining the envelope of any signal, be it narrow or
wide band.

In wide band contexts, the problem of applying a Hilbert transform is made 
particularly difficult because of edge effects.  A narrow band Hilbert transform
can be much more readily constructed as either a FIR or IIR filter.

\section{Eye based Windowing}
The idea presented here is remarkably simple and effective.  We forego any 
idea of windowing functions, such as the Hamming or Blackman windows, and
instead we simply implement a Hilbert transform by 
\begin{enumerate}
	\item Using a sufficiently large FFT, let's call this $L$.
	\item Using a hop size $H$ small enough ($H < L$)so that centering
		$H$ in $L$ removes edge effects.
	\item If necessary, padding the input signal so that advancing it 
	by $H$ samples at a time will center $H$ samples in $L$.
\end{enumerate}

What is perhaps unusual here, is that we do not need, nor do we {\em want}
to restrict the FFT to the data being analyzed.  Instead, in the context 
of using a Hilbert transform, we desire continuity of the amplitude envelope
and phase differences, which an only really be achieved for a segment of a 
signal with both preceding and succeeding context.


\section{}
\end{document}
